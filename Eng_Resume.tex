% LaTeX resume using res.cls
\documentclass[line,margin]{res}
% \usepackage{helvetica} % uses helvetica postscript font (download helvetica.sty)
% \usepackage{newcent}   % uses new century schoolbook postscript font

\begin{document}

\name{NATHAN T. HUI}
% \address used twice to have two lines of address
\address{8434 Via Sonoma, \#63, La Jolla, CA 92037}
\address{nthui@eng.ucsd.edu, (408) 780-8594}


\begin{resume}

\section{OBJECTIVE}
	% Engineering team leader looking to get involved with multidisciplinary and high impact projects.
	Electrical engineer looking to explore hardware and software design with a focus on system design and integration.
	% Electrical engineer looking to explore embedded systems and robotics, particularly embedded control systems, embedded sensor networks, and low-level data aggregation from the perspective of developing cutting edge technologies and applications.
	% Electrial engineer looking to explore embedded systems and robotics, particularly embedded control systems, embedded sensor networks, and low-level data aggregation from the perspective of developing technologies to enable scientific research.
	%Looking for an engineering position in researching, designing, and building autonomous vehicles in a challenging and dynamic team environment.
	%Looking for a position working with groups in the outdoors with a focus in safety and/or search and rescue.


\section{ENGINEERING EXPERIENCE}

	{\sl UCSD E4E / CISA3 - Project Lead / Graduate Researcher} \hfill Winter 2014 - current
	\begin{itemize}
		\item Leading and managing the development, testing, and deployment of C/C++ and Python embedded software and computing hardware for aerial wildlife radio tracking surveys using drones, using software-defined radios to detect radio trackers and embedded Linux single-board computers for online processing and tracking.
		\item Developed and tested 3D printed mechanical subsystems in Solidworks for mounting scientific payloads to multirotor aircraft.
		% \item Conducted daytime and nighttime test flights and scientific survey flights of a multirotor-based wildlife tracking system using radio transmitters.
		\item Analyzed post-crash telemetry in deployed and testing enviornments to determine and mitigate causes of crashes to facilitate drone system robustness.
		\item Developed and deployed embedded C software onto an ATMega microprocessor for controlling a sensor payload and aggregating data from low-level sensors over serial communications.
		% \item Exploring technology to dynamically generate and optimize drone flight paths for aerial survey using real-time feedback from cameras and other drone sensors, including terrain following/mapping and source seeking techniques.
		\item Developed integrated 3-D visual mapping tools using stereo cameras to enable researchers to make high-quality 3-D models of areas in near real-time under an Undergraduate Research Fellowship from the Center of Integrated Access Networks (CIAN) in 2015.
		% \item Developed a stabilization controller in C on an embedded microcontroller for an actively stabilized DSLR mount for a persistent aerial survey platform, aimed at providing researchers with a wide-angle view coupled with pan-capable zoom imagery.
		% \item Mentoring and managing students working on various research projects.
		% \item Primary Investigator: Ryan Kastner, (858) 534-3534, kastner@ucsd.edu; Curt Schurgers, (858) 246-1442, cschurgers@ucsd.edu
	\end{itemize}

	{\sl UCSD SIO Jaffe Lab - Graduate Researcher} \hfill Spring 2018 - Summer 2018
	\begin{itemize}
		\item Utilized Python to develop a distributed control system for a hemispherical multi-camera array run by a cluster of Raspberry Pi Zeros, and created a workflow to synchronize and stitch immersive video from the camera array.
		% \item Improved the design of the structural support for the multi-camera array in Solidworks to improve stability and rigidness.
		% \item Supported a deployment of a plankton camera on a Conductivity, Temperature, and Depth sensor rig on the R/V Robert Gordon Sproul.
	\end{itemize}

	{\sl UCSD Veridrone - Graduate Researcher} \hfill Spring 2017 - Spring 2018
	\begin{itemize}
		\item Mentored an undergraduate research project examining using machine learning to predict multirotor failures by examining real-time telemetry.
		\item Performed state space exploration of slow discrete time implementation of source seeking algorithms.
		% \item Primary Investigators: Ryan Kastner, (858) 534-3534, kastner@ucsd.edu; Sorin Lerner, (858) 534-8883, lerner@ucsd.edu; Miroslav Krstic, (858) 534-5556, mkrstic@ucsd.edu
	\end{itemize}

	% {\sl UCSD AUVSI - System Lead} \hfill Fall 2012 - Spring 2014
	% \begin{itemize}
	% 	% \item Led development of an actively stabilized composite DSLR gimbal, composite payload computer enclosure, composite payload launcher, composite battery carrier, and autopilot enclosure with minimized system volume and weight.
	% 	\item Developed embedded C software on a 32-bit ARM microcontroller to aggregate telemetry from an autopilot, camera gimbal, and imaging system and forward the summarized data to an image processing pipeline.
	% 	% \item Designed and implemented embedded C software on an ATMega microprocessor to provide control link failsafe control.
	% 	% \item Developed and validated embedded software on an 8-bit Atmel microcontroller in C to analyze and generate control signals for use in an RC failsafe device.
	% 	% \item Developed a DSLR gimbal using SolidWorks and various machine tools for isolating the imaging system from airframe movement during flight.
	% \end{itemize}

% \section{WORK EXPERIENCE}
% 	{\sl Biosero LLC - Engineering Associate} \hfill Summer 2013
% 	\begin{itemize}
% 		\item Optimized tube cap removal device performance, and conducted systems integration, formalizing system specifications and integration with company automation products.
% 		\item Designed low-cost automated plate handling system using SolidWorks for coupling two automated liquid handling enclosures.
% 		\item Developed extensible code structure for software integration with touch screens, permitting intuitive user interface with automated liquid handling enclosures.
% 		\item Leveraged statistical computation capabilities of Microsoft Excel to analyze accuracy and precision of automated syringe pumps.
% 		% \item Supervisor: Alice Tanibata, (858) 525-1645, alicetanibata@biosero.com
% 	\end{itemize}
\section{EDUCATION}
	{\sl Master of Science,} Electrical Engineering \hfill September 2019\\
	Focus in Intelligent Systems, Robotics, and Controls \hfill GPA: 3.005 / 4.000\\
	UC San Diego, San Diego, CA\\

	{\sl Bachelor of Science,} Electrical Engineering  \hfill March 2017\\
    UC San Diego, San Diego, CA \hfill GPA: 3.092 / 4.000\\

\section{PORTFOLIO}
	{\sl Linkedin} - https://www.linkedin.com/in/ntlhui

	{\sl Github} - https://github.com/ntlhui

\end{resume}
\end{document}

